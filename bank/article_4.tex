\documentclass[
10pt, % Main document font size
letterpaper, % Paper type, use 'letterpaper' for US Letter paper
oneside, % One page layout (no page indentation)
%twoside, % Two page layout (page indentation for binding and different headers)
headinclude,footinclude, % Extra spacing for the header and footer
BCOR5mm, % Binding correction
]{scrartcl}

\usepackage[spanish]{babel}

%%%%%%%%%%%%%%%%%%%%%%%%%%%%%%%%%%%%%%%%%
% Arsclassica Article
% Structure Specification File
%
% This file has been downloaded from:
% http://www.LaTeXTemplates.com
%
% Original author:
% Lorenzo Pantieri (http://www.lorenzopantieri.net) with extensive modifications by:
% Vel (vel@latextemplates.com)
%
% License:
% CC BY-NC-SA 3.0 (http://creativecommons.org/licenses/by-nc-sa/3.0/)
%
%%%%%%%%%%%%%%%%%%%%%%%%%%%%%%%%%%%%%%%%%

%----------------------------------------------------------------------------------------
%	REQUIRED PACKAGES
%----------------------------------------------------------------------------------------

\usepackage[
nochapters, % Turn off chapters since this is an article        
beramono, % Use the Bera Mono font for monospaced text (\texttt)
eulermath,% Use the Euler font for mathematics
pdfspacing, % Makes use of pdftex’ letter spacing capabilities via the microtype package
dottedtoc % Dotted lines leading to the page numbers in the table of contents
]{classicthesis} % The layout is based on the Classic Thesis style

\usepackage{arsclassica} % Modifies the Classic Thesis package

\usepackage[T1]{fontenc} % Use 8-bit encoding that has 256 glyphs

\usepackage[utf8]{inputenc} % Required for including letters with accents

\usepackage{graphicx} % Required for including images
\graphicspath{{Figures/}} % Set the default folder for images

\usepackage{enumitem} % Required for manipulating the whitespace between and within lists

\usepackage{lipsum} % Used for inserting dummy 'Lorem ipsum' text into the template

\usepackage{subfig} % Required for creating figures with multiple parts (subfigures)

\usepackage{amsmath,amssymb,amsthm} % For including math equations, theorems, symbols, etc

\usepackage{varioref} % More descriptive referencing

%----------------------------------------------------------------------------------------
%	THEOREM STYLES
%---------------------------------------------------------------------------------------

\theoremstyle{definition} % Define theorem styles here based on the definition style (used for definitions and examples)
\newtheorem{definition}{Definition}

\theoremstyle{plain} % Define theorem styles here based on the plain style (used for theorems, lemmas, propositions)
\newtheorem{theorem}{Theorem}

\theoremstyle{remark} % Define theorem styles here based on the remark style (used for remarks and notes)

%----------------------------------------------------------------------------------------
%	HYPERLINKS
%---------------------------------------------------------------------------------------

\hypersetup{
%draft, % Uncomment to remove all links (useful for printing in black and white)
colorlinks=true, breaklinks=true, bookmarks=true,bookmarksnumbered,
urlcolor=webbrown, linkcolor=RoyalBlue, citecolor=webgreen, % Link colors
pdftitle={}, % PDF title
pdfauthor={\textcopyright}, % PDF Author
pdfsubject={}, % PDF Subject
pdfkeywords={}, % PDF Keywords
pdfcreator={pdfLaTeX}, % PDF Creator
pdfproducer={LaTeX with hyperref and ClassicThesis} % PDF producer
} 
\hyphenation{Fortran hy-phen-ation}

\title{\normalfont\spacedallcaps{Banco de Datos INMEGEN}}
\author{\spacedlowsmallcaps{Rodrigo García Herrera}}
\date{} % An optional date to appear under the author(s)



\begin{document}

%----------------------------------------------------------------------------------------
%	HEADERS
%----------------------------------------------------------------------------------------

\renewcommand{\sectionmark}[1]{\markright{\spacedlowsmallcaps{#1}}} % The header for all pages (oneside) or for even pages (twoside)
%\renewcommand{\subsectionmark}[1]{\markright{\thesubsection~#1}} % Uncomment when using the twoside option - this modifies the header on odd pages
\lehead{\mbox{\llap{\small\thepage\kern1em\color{halfgray} \vline}\color{halfgray}\hspace{0.5em}\rightmark\hfil}} % The header style

\pagestyle{scrheadings} % Enable the headers specified in this block


%----------------------------------------------------------------------------------------
%	TABLE OF CONTENTS & LISTS OF FIGURES AND TABLES
%----------------------------------------------------------------------------------------

\maketitle % Print the title/author/date block
\setcounter{tocdepth}{2} % Set the depth of the table of contents to show sections and subsections only
\tableofcontents % Print the table of contents
%\listoffigures % Print the list of figures
%\listoftables % Print the list of tables



%%%%%%%%%%%%
% The text %
%%%%%%%%%%%%

\section{Antecedentes}


\subsection{Open Science, Open Data}

Los datos constituyen la evidencia del conocimiento científico. Entre
más datos disponibles mayor nivel de transparencia y reproducibilidad,
luego más eficiencia en el proceso de generación de conocimiento
científico. \cite{molloy_open_2011}.


\cite{walport_sharing_2011}
\cite{piwowar_sharing_2007}
\cite{alsheikh-ali_public_2011}
\cite{wicherts_poor_2006}
\cite{wicherts_publish_2012}
\cite{wicherts_willingness_2011}

\subsection{Datos en el INMEGEN}


\section{Objetivos y alcance}


\subsection{Banco de datos}

Diseño e implementacion de un sistema informático que permita a
investigadores del INMEGEN compartir datos de manera uniforme etc. con
la comunidad científica. Que permita a la comunidad científica obtener
datos publicados por el instituto. Que ayude a establecer vínculos y a
medir el impacto de los datos. Que cumpla con estándares.
\cite{_data_????}
\cite{altman_proposed_2007}


\subsection{Interfaces Global Alliance}

Matchmaker Exchange: a federated network of rare disease data sets.

BRCA Challenge.

Beacon Project.

Reads API


\subsection{Enlaces con otros Repositorios}
Establecer un proceso que permita identificar repositorios públicos
con mayor masa crítica para ciertos tipos de datos.

Establecer un proceso que permita identificar cuáles de nuestros datos
van en esos repositorios.
\cite{_genebank_????}
\cite{king_introduction_2007}

Gene expression omnibus

\section{Justificación}

\subsection{Publicación de datos.}

\subsubsection{Requisito de Revistas}
Tener un banco de datos institucional y mecanismos institucionales
para la publicación de datos en repositorios establecidos facilitará
el cumplimiento del requisito de publicación de datos que
habitualmente es política \cite{hrynaszkiewicz} de estas revistas
donde publicamos.


\subsubsection{Reproducibilidad, transparencia}
Ciencia reproducible es de mayor calidad y redunda en más exposición y
citación. \cite{piwowar_sharing_2007} \cite{ioannidis}

\subsection{Vinculación, colaboración}
Compartir datos da ocasión de establecer contacto con organizaciones e
individuos interesados en las mismas líneas de investigación.

Algunos tipos de investigación ganan gran poder estadístico con la
inclusión de más pacientes, mendelian, (citar acá global alliance
mendelian). 


\section{Metodología}
Reference to Figure~\vref{fig:gallery}. % The \vref command specifies the location of the reference

% \begin{figure}[tb]
% \centering 
% \includegraphics[width=0.5\columnwidth]{GalleriaStampe} 
% \caption[An example of a floating figure]{An example of a floating figure (a reproduction from the \emph{Gallery of prints}, M.~Escher,\index{Escher, M.~C.} from \url{http://www.mcescher.com/}).} % The text in the square bracket is the caption for the list of figures while the text in the curly brackets is the figure caption
% \label{fig:gallery} 
% \end{figure}


\subsection{Creación de Grupo de trabajo}
(aca citar data.gob.mx)

\subsection{Inventario Institucional}
\cite{_nih_????}

\subsection{Publicación}


\subsection{Promoción}
\cite{schofield_post-publication_2009}


%----------------------------------------------------------------------------------------
%	BIBLIOGRAPHY
%----------------------------------------------------------------------------------------

\renewcommand{\refname}{\spacedlowsmallcaps{References}} % For modifying the bibliography heading

\bibliographystyle{unsrt}

%\bibliography{sample.bib}
\bibliography{inmegen_databank}

%----------------------------------------------------------------------------------------

\end{document}